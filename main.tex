\documentclass{article}

\usepackage{amsmath}
\usepackage{siunitx}

\author{Oskar Mampe}
\title{Quaternions}
\date{\today}

\begin{document}

\maketitle
\section*{Problem}
An artist has given you two keyframe rotation matrices of an object.  At time step $0$, the artist wants the object rotated by \ang{45} CCW around vector $(2,1,4)$ and at time step $100$, the object should be rotated by \ang{123} CW around vector $(0,1,1)$.
Use quaternions to determine the axis and angle of rotation required to achieve this rotation in $100$ steps.  Show all work.\\[1em]

\textit{All calculations are rounded to 5 decimal places.}

\section*{Solution}



A vector $\vec{v} = (2,1,4)$ is formed using the information given above. Also, the angle $\ang{45}$ CCW becomes $\theta_0 = \frac{45}{2} = 22.5$. 
\begin{align*}
q_0 &= (cos(\theta_0), sin(\theta_0) \times \vec{|v|}))\\[1em]
||\vec{v}|| &= \sqrt{2^2 + 1^ 2 + 4^2}
= \sqrt{21} 
= 4.58258\\[1em]
%
|\vec{v}| &= 
\frac{\vec{v}}{||\vec{v}||} = 
\frac{(2, 1, 4)}{4.58258 }
= (0.43644, 0.21822, 0.87287)\\[1em]
%
cos&(\theta_0) = 0.92388\\[1em]
sin&(\theta_0) = 0.38268\\[1em]
q_0 &= (0.92388,0.38268 \times 0.43644, 0.38268 \times 0.21822, 0.38268 \times 0.87287)\\[1em]
q_0 &= 0.92388 + 0.16702i + 0.08351j + 0.33403k\\[1em]
\end{align*}

A vector $\vec{w} = (0,1,1)$ is formed using the information given above. Also, the angle $\ang{123}$ CW becomes $\theta_{100} = -\frac{123}{2} = -61.5$.

\begin{align*}
q_{100} &= (cos(\theta_{100}), sin(\theta_{100}) \times \vec{|w|}))\\[1em]
||\vec{w}|| &= \sqrt{0^2 + 1^2 + 1^2} = \sqrt{2} = 1.41421\\[1em]
|\vec{w}| &= \frac{\vec{w}}{||\vec{w}||} = \frac{(0,1,1)}{1.41421} = (0, 0.70711, 0.70711)\\[1em]
\\[1em]
cos&(\theta_{100}) = 0.47716\\[1em]
sin&(\theta_{100}) = -0.87882\\[1em]
q_{100} &= (0.47716, 0, -0.87882 \times 0.70711, -0.87882 \times 0.70711)\\[1em]
q_{100} &= 0.47716 + 0i -0.62142j -0.62142k\\[1em]
\end{align*}

Next, to find the quaternion required $q_{\delta}$, the quaternion $q_{\Delta}$ needs to be calculated by multiplying quaternions $q_{100}$ and the inverse of $q_{0}$. This is because $q_{100}pq^{-1}_{100} = q_{\Delta}(q_{0}pq^{-1}_{0})q_{\Delta}^{-1}$. This can be simplified to $q_{\Delta}q_0 = q_{100}$, therefore $q_{\Delta} = q_{100}q^{-1}$. Since $q_{0}$ is a unit quaternion, its magnitude is $1$, so only taking the conjugate $q^*_0$ is required, as $q^{-1}_0 = \frac{q^*_0}{||q_0||} = \frac{q^*_0}{1} = q^*_0$.

\begin{align*}
q_0^* &= 0.92388 - 0.16702i - 0.08351j - 0.33403k\\[1em]
q_{100} &= 0.47716 + 0i - 0.62142j - 0.62142k\\[1em]
q_{\Delta} &= q_0^* \times q_{100}\\[1em]
q_{\Delta} &= 0.44084 + 0i - 0.57412j - 0.57412k\\
 &- 0.07970i - 0i^2 + 0.10379ij + 0.10379ik\\
 &- 0.03985j - 0ji + 0.05189j^2 + 0.05189jk\\
 &- 0.15939k - 0ki + 0.20757kj + 0.20757k^2\\[1em]
 %
 &= 0.44084 - 0.57412j - 0.57412k - 0.07970i\\
 &+ 0.10379k - 0.10379j - 0.03985j - 0.05189\\
 &+ 0.05189i - 0.15939k - 0.20757i - 0.20757\\[1em]
q_{\Delta} &=0.18138 - 0.23538i - 0.71776j - 0.62972k\\[1em]
\end{align*}

Now that $q_{\Delta}$ is found, $\theta_{\Delta}$ and $\vec{v}_{\Delta}$ can be found. Then the quaternion $q_{\delta}$ can be formed by using $\frac{\theta_{\Delta}}{100}$ and $\vec{v}_{\Delta}$.
\begin{align*}
(q_{\delta})^{100} &= q_{\Delta}\\[1em]
\theta_{\delta} &= \frac{\theta_{\Delta}}{100}\\[1em]
\theta_{\Delta} &= arccos(0.18138) = 79.54985\\[1em]
\theta_{\delta} &= \frac{79.54985}{100}\\[1em]
\theta_{\delta} &= 0.79550\\[1em]
sin&(79.54985) = 0.98341\\[1em]
\vec{v}_{\delta} &=  (- \frac{0.23538}{0.98341}, - \frac{0.71776}{0.98341}, - \frac{0.62972}{0.98341})\\
\vec{v}_{\delta} &= (-0.23935, -0.72987, -0.64034)\\[1em]
\end{align*}

Finally, the quaternion $q_{\delta}$ with the required angle and axis can be formed:

\begin{align*}
cos&(0.79550) = 0.99990\\[1em]
sin&(0.79550) = 0.01388\\[1em]
q_{\delta} &= (0.99990, 0.01388 \times - 0.23935, 0.01388 \times - 0.72987, 0.01388 \times - 0.64034)\\[1em]
q_{\delta} &= 0.99990 - 0.00332i - 0.01013j - 0.00889k\\
\end{align*}

\end{document}